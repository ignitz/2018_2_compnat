\section{Experimentos}

O código foi executado com diferentes parâmetros para \textbf{graph1} e \textbf{graph2}.
Foram eles:
\begin{itemize}
  \item nº de formigas = $\{ 50,100,200,1000\}$
  \item iterações = $\{ 100\}$ 
  \item $\sigma$ = $\{ 0.1,0.3,0.5,0.7,0.9\}$
  \item $\alpha$ = $\{ 0.1,0.5,1.0,2.0\}$
  \item $\beta$ = $\{ 0.1,0.5,1.0,2.0\}$
  \item elitismo = $\{ 0,1,5,10,15\}$
\end{itemize}

Todos os dados dos experimentos estão disponíveis no:

\url{https://homepages.dcc.ufmg.br/~yuriniitsuma/natcomp/}

\begin{itemize}
  \item \textbf{out.tar.gz} contendo as saídas com cada combinação de parâmetros.  
  \item \textbf{All\_plots.pdf} contendo todos os gráficos (3252) de todos os experimentos.  
  Cada gráfico é a média das saídas de cada combinação dos parâmetros.
\end{itemize}

No gráfico cada representa:

\begin{itemize}
  \item \textbf{Azul}: melhor fitness
  \item \textbf{Verde}: média das fitness
  \item \textbf{Laranja}: pior fitness
  \item \textbf{Vermelha}: solução ótima
\end{itemize}

Destacarei os gráficos que representam as análises mais significantes do experimento.

\subsection{\textbf{graph2}}

\begin{figure}[H]
  \centering
  \includegraphics[width=0.5\textwidth]{../img/graph2_Iterations_100_NumberOfAnts_50_Alpha_1.0_Beta_1.0_Sigma_0.1_Elitism_0.pdf}
\end{figure}

\subsection*{Variação da quantidade de formigas}

\begin{figure}[H]
  \centering
  \includegraphics[width=0.5\textwidth]{../img/graph2_Iterations_100_NumberOfAnts_100_Alpha_1.0_Beta_1.0_Sigma_0.1_Elitism_0.pdf}
\end{figure}

\begin{figure}[H]
  \centering
  \includegraphics[width=0.5\textwidth]{../img/graph2_Iterations_100_NumberOfAnts_200_Alpha_1.0_Beta_1.0_Sigma_0.1_Elitism_0.pdf}
\end{figure}

\begin{figure}[H]
  \centering
  \includegraphics[width=0.5\textwidth]{../img/graph2_Iterations_100_NumberOfAnts_1000_Alpha_1.0_Beta_1.0_Sigma_0.1_Elitism_0.pdf}
\end{figure}

Percebe-se um efeito colateral dado a quantidade de formigas excessivas, pois os caminhos são aleatórios tendo uma pequena chance
da convergência ser muito lenta. Manteremos $100$ formigas e analisaremos a variação de $\alpha$.

\subsection*{Variação de $\alpha$}

\begin{figure}[H]
  \centering
  \includegraphics[width=0.5\textwidth]{../img/graph2_Iterations_100_NumberOfAnts_100_Alpha_0.1_Beta_1.0_Sigma_0.1_Elitism_0.pdf}
\end{figure}

\begin{figure}[H]
  \centering
  \includegraphics[width=0.5\textwidth]{../img/graph2_Iterations_100_NumberOfAnts_100_Alpha_0.5_Beta_1.0_Sigma_0.1_Elitism_0.pdf}
\end{figure}

\begin{figure}[H]
  \centering
  \includegraphics[width=0.5\textwidth]{../img/graph2_Iterations_100_NumberOfAnts_100_Alpha_1.0_Beta_1.0_Sigma_0.1_Elitism_0.pdf}
\end{figure}

\begin{figure}[H]
  \centering
  \includegraphics[width=0.5\textwidth]{../img/graph2_Iterations_100_NumberOfAnts_100_Alpha_2.0_Beta_1.0_Sigma_0.1_Elitism_0.pdf}
\end{figure}

Verificamos que aumenta a taxa de convergência, mas um valor maior aumenta a convergência prematura.  

\subsection*{Variação de $\beta$}

\begin{figure}[H]
  \centering
  \includegraphics[width=0.5\textwidth]{../img/graph2_Iterations_100_NumberOfAnts_100_Alpha_1.0_Beta_0.1_Sigma_0.1_Elitism_0.pdf}
\end{figure}

\begin{figure}[H]
  \centering
  \includegraphics[width=0.5\textwidth]{../img/graph2_Iterations_100_NumberOfAnts_100_Alpha_1.0_Beta_0.5_Sigma_0.1_Elitism_0.pdf}
\end{figure}

\begin{figure}[H]
  \centering
  \includegraphics[width=0.5\textwidth]{../img/graph2_Iterations_100_NumberOfAnts_100_Alpha_1.0_Beta_1.0_Sigma_0.1_Elitism_0.pdf}
\end{figure}

\begin{figure}[H]
  \centering
  \includegraphics[width=0.5\textwidth]{../img/graph2_Iterations_100_NumberOfAnts_100_Alpha_1.0_Beta_2.0_Sigma_0.1_Elitism_0.pdf}
\end{figure}

Tem efeitos mais significativos mas não foi encontrado ainda um efeito colateral. Vamos agora variar a evaporação.

\subsection*{Variação de $\sigma$}

\begin{figure}[H]
  \centering
  \includegraphics[width=0.5\textwidth]{../img/graph2_Iterations_100_NumberOfAnts_100_Alpha_1.0_Beta_1.0_Sigma_0.1_Elitism_0.pdf}
\end{figure}

\begin{figure}[H]
  \centering
  \includegraphics[width=0.5\textwidth]{../img/graph2_Iterations_100_NumberOfAnts_100_Alpha_1.0_Beta_1.0_Sigma_0.3_Elitism_0.pdf}
\end{figure}

\begin{figure}[H]
  \centering
  \includegraphics[width=0.5\textwidth]{../img/graph2_Iterations_100_NumberOfAnts_100_Alpha_1.0_Beta_1.0_Sigma_0.5_Elitism_0.pdf}
\end{figure}

\begin{figure}[H]
  \centering
  \includegraphics[width=0.5\textwidth]{../img/graph2_Iterations_100_NumberOfAnts_100_Alpha_1.0_Beta_1.0_Sigma_0.7_Elitism_0.pdf}
\end{figure}

\begin{figure}[H]
  \centering
  \includegraphics[width=0.5\textwidth]{../img/graph2_Iterations_100_NumberOfAnts_100_Alpha_1.0_Beta_1.0_Sigma_0.9_Elitism_0.pdf}
\end{figure}

Aqui percebe-se um bom ganho das soluções mas que valores muito alto possui efeito colateral que forçam
as melhores soluções e a pior convergirem para a média fazendo um efeito sanduiche. Agora vamos analisar
o efeito do parâmetro de elitismo.

\subsection*{Variação do parâmetro de elitismo}

\begin{figure}[H]
  \centering
  \includegraphics[width=0.5\textwidth]{../img/graph2_Iterations_100_NumberOfAnts_100_Alpha_1.0_Beta_1.0_Sigma_0.7_Elitism_1.pdf}
\end{figure}

\begin{figure}[H]
  \centering
  \includegraphics[width=0.5\textwidth]{../img/graph2_Iterations_100_NumberOfAnts_100_Alpha_1.0_Beta_1.0_Sigma_0.7_Elitism_5.pdf}
\end{figure}

\begin{figure}[H]
  \centering
  \includegraphics[width=0.5\textwidth]{../img/graph2_Iterations_100_NumberOfAnts_100_Alpha_1.0_Beta_1.0_Sigma_0.7_Elitism_10.pdf}
\end{figure}

\begin{figure}[H]
  \centering
  \includegraphics[width=0.5\textwidth]{../img/graph2_Iterations_100_NumberOfAnts_100_Alpha_1.0_Beta_1.0_Sigma_0.7_Elitism_15.pdf}
\end{figure}

Percebe-se que a melhor solução converge junto com a média mas ainda mantém alguma variabilidade.

\subsection{\textbf{graph1}}

No \textit{graph1} percebemos que o parâmetro $\sigma$ é o mais significativo, talvez por alguma característica no grafo.

\begin{figure}[H]
  \centering
  \includegraphics[width=0.5\textwidth]{../img/graph1_Iterations_100_NumberOfAnts_100_Alpha_1.0_Beta_0.5_Sigma_0.9_Elitism_10.pdf}
\end{figure}

\begin{figure}[H]
  \centering
  \includegraphics[width=0.5\textwidth]{../img/graph1_Iterations_100_NumberOfAnts_100_Alpha_1.0_Beta_0.5_Sigma_0.3_Elitism_10.pdf}
\end{figure}

Valores muitos altos de evaporação ocasionam numa maior variabilidade talvez pela quantidade alta de arestas
afetam algum efeito de inanição e a aresta acaba sendo inutilizado.

\subsection{\textbf{graph3}}

No \textbf{graph3}, os parâmetros que influenciaram na melhor solução foi $\beta$ e a quantidade máxima de formigas
pelo fato do grafo ser mais esparso que os dois anteriores.

As melhores soluções, da melhor formiga, foram:

\[\{ 9786.16,9778.75,9773.58\} \]

\begin{figure}[H]
  \centering
  \includegraphics[width=0.5\textwidth]{../img/graph3_Iterations_100_NumberOfAnts_50_Alpha_1.0_Beta_2.0_Sigma_0.1_Elitism_1.pdf}
  \caption{9786.16}
\end{figure}

\begin{figure}[H]
  \centering
  \includegraphics[width=0.5\textwidth]{../img/graph3_Iterations_100_NumberOfAnts_1000_Alpha_2.0_Beta_2.0_Sigma_0.9_Elitism_10.pdf}
  \caption{9778.75}
\end{figure}

\begin{figure}[H]
  \centering
  \includegraphics[width=0.5\textwidth]{../img/graph3_Iterations_100_NumberOfAnts_50_Alpha_1.0_Beta_2.0_Sigma_0.1_Elitism_0.pdf}
  \caption{9773.58}
\end{figure}
