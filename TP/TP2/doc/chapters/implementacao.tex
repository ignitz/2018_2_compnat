\section{Implementação}

O ACO foi modelado utilizando as principais bibliotecas:

\begin{itemize}
	\item \textbf{numpy}
	\item \textbf{networkx}: biblioteca para grafos
\end{itemize}

As bibliotecas instaladas, via \textbf{pip}, pelo \textbf{virtual-env} encontra-se
no arquivo \textbf{req.txt}.

Possui características de um ACO clássico do problema do caixeiro viajante.
A única diferênça ao modelo deste trabalho é a atratividade do caminho $\eta_{xy}$
e taxa de depósito de feromônio nas arestas $\Delta\tau_{xy}^k$ que serão
descritos nas seções a seguir. 

\subsection{Grafo}

Seja o grafo $G(V, E)$, definimos baseado na especificação.

\begin{itemize}
	\item O grafo $G$ é direcionado.
	\item Cara aresta do grafo contém, como atributo, pesos dados como entrada, a taxa de feromônio $t_{xy}$ e a probabilidade
	da formiga escolher a aresta $p_{xy}$ que é atualizado no final de cada iteração.  
    Sendo $xy$ a direção da aresta do nó $x$ a $y$.
    \item A taxa de feromônio inicial das arestas é $0.1$.
	\item Como o custo de encontrar um caminho entre os nós $u$ e $v$ é caro, assumimos que sempre existe um caminho simples do nó $1$ ao $n$.
    Caso contrário,
    entrará em loop infinito. Isto não acontece com os \textit{datasets} \textbf{graph1}, \textbf{graph2} e \textbf{graph3}.
    \item O algoritmo não gera soluções inválidas, se o caminho gerado for inválido, a formiga é descartada e gerada novamente.
\end{itemize}

\subsection{Elitismo}

Foi utilizado um parâmetro de elitismo que mantém os $k$ melhores formigas/caminhos
para a próxima iteração.

\subsection{Escolha da Aresta}

A formiga escolhe a aresta $xy$ pela seguinte fórmula de probabilidade.

$$p_{xy} = \frac{(\tau_{xy})^\alpha(\eta_{xy})^\beta}{\sum\limits_{e_{xj} \in E} (\tau_{xj})^\alpha(\eta_{xj})^\beta }$$

O denominador representa a soma entre todas as arestas vizinhas e $x$.

O termo $\eta_{xy}$ representa a atratividade do caminho e foi definido no modelo como ${\eta _{xy}} = w(x,y)$ em que $w$ é peso da aresta do nó $x$ para $y$.

\subsection{Fitness}

A fitness da formiga é o custo do caminho do vértice $1$ a $n$.

$$fit(P^k) = \sum\limits_{i} w(e_i^k)$$

\subsection{Atualização Feromônio}

A atualização do feromônio nas arestas ocorre após a construção do grafo ou ao final de cada iteração.

Dado que:

\begin{itemize}
    \item $\tau_{xy}$ é a taxa de feromônio depositado na aresta
    \item $\sigma$ é a taxa de evaporação do feromônio $(0 < \sigma < 1)$.
    \item $\Delta \tau _{xy}^k$ é a  quantidade de feromônio que a formiga $k$ irá depositar na aresta $xy$.
\end{itemize}

A taxa de depósito é:

\[\Delta \tau _{xy}^k = \frac{{fit({P^k}) - fit(\{ e_1^k,e_2^k, \ldots ,e_{xy}^k\} ) + 0.5}}{{\max \left( {fit({P^k});{\rm{for\ all\ ants}}} \right)}}\]

Assim o feromônio é mais forte no início do caminho e fica mais fraco ao final. Isto ajuda
a criar mais caminhos com o prefixo do melhor caminho e tentar variar nas arestas posteriores.
Também sofre uma normalização dividindo pela fitness máximo entre todas as formigas, assim penalizando
as formigas com fitness mais baixas.

A atualização da taxa de feromônio e dada por:

\[{\tau _{xy}} \leftarrow \min \left( {100,(1 - \sigma ){\tau _{xy}} + \sum\limits_k \Delta  \tau _{xy}^k} \right)\]

Observe que há um teto de valor $100$ para evitar que o valor exploda e que matenha uma certa variabilidade dos caminhos. 
